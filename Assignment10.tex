%\documentclass[11pt,reqno]{amsart}
\documentclass[11pt,reqno]{article}
\usepackage[margin=.8in, paperwidth=8.5in, paperheight=11in]{geometry}
%\usepackage{geometry}                % See geometry.pdf to learn the layout options. There are lots.
%\geometry{letterpaper}                   % ... or a4paper or a5paper or ... 
%\geometry{landscape}                % Activate for for rotated page geometry
%\usepackage[parfill]{parskip}    % Activate to begin paragraphs with an empty line rather than an indent7
\usepackage{graphicx}
\usepackage{pstricks}
\usepackage{amssymb}
\usepackage{epstopdf}
\usepackage{amsmath}
\usepackage{subfigure}
\usepackage{caption}
\pagestyle{plain}
%\renewcommand{\topfraction}{0.3}
%\renewcommand{\bottomfraction}{0.8}
%\renewcommand{\textfraction}{0.07}
\DeclareGraphicsRule{.tif}{png}{.png}{`convert #1 `dirname #1`/`basename #1 .tif`.png}

\title{Real Analysis $\mathbb{I}$: \\ Assignment 10}
\author{Andrew Rickert}
\date{Started: January 4, 2011 \\ \hspace{1pt} Ended: January ??,  2011}                                           % Activate to display a given date or no date

\begin{document}
\maketitle


% Page 1
\begin{flushleft} 
\textbf{Class 18.100B} - Problem 1\\
\rule{500pt}{1pt}\\
\end{flushleft} 

\noindent First, we want to show that a sequence of functions defined by 
\[ f_0(t) = \sin t \quad \text{and} \quad f_{n+1}(t) = \frac{2}{3}f_n(t) + 1\]
is such that $\lim_{n \to \infty} f_n(t) \to 3$ uniformly on $\mathbb{R}$. Lastly we want to know what happens in the case that $f_0(t) = t^2$.\\
\indent We answer both questions by considering the mapping $T(x) = \frac{2}{3}x + 1$. If we define $T_1(x) = T(x)$ and $T_{n+1}(x) = \frac{2}{3}T_n(x) + 1$ then we find that
\begin{equation}
T_n(x) = \left( \frac{2}{3} \right)^n x + \sum_{k=0}^{n-1} \left( \frac{2}{3} \right)^k \quad \text{for $n \ge 1$} \label{eqn:Tfunction}
\end{equation}
This can be shown inductively. For $n=1$ we have $T_1 =  \left( \frac{2}{3} \right)^1 x +  \left( \frac{2}{3} \right)^0 =  \frac{2}{3} x + 1$ which establishes the induction base. Now we assume the formula ($\ref{eqn:Tfunction}$) is true for $T_n$ and calculate as follows:
\begin{eqnarray*}
T_{n+1}(x) &=& \frac{2}{3}T_n(x) + 1 \\
		 &=& \frac{2}{3}\left( \left( \frac{2}{3} \right)^n x + \sum_{k=0}^{n-1} \left( \frac{2}{3} \right)^k \right) + 1 \\
		 &=& \left( \frac{2}{3} \right)^{n+1} x + \sum_{k=1}^{n} \left( \frac{2}{3} \right)^k + 1 \\
		 &=& \left( \frac{2}{3} \right)^{n+1} x + \sum_{k=0}^{n} \left( \frac{2}{3} \right)^k\\
\end{eqnarray*}
So the formula ($\ref{eqn:Tfunction}$) is true for $n+1$ which completes the induction.\\
We now let $x = \sin t$ and get $f_n(t) = T_n(\sin t)$. We choose $N_1$ such that $|\sum_{k=0}^{n-1} \left( \frac{2}{3} \right)^k - 3| < \frac{\epsilon}{2}$ for $n > N_1$ and $N_2$ such that $|\left( \frac{2}{3} \right)^n| < \frac{\epsilon}{2}$ for $n > N_2$. If we let $N = $ Max$(N_1, N_2)$ we may calculate as follows:
\begin{eqnarray*}
|f_n(t) - 3| &=& |\left( \frac{2}{3} \right)^n \sin t + \sum_{k=0}^{n-1} \left( \frac{2}{3} \right)^k - 3| \\
		&\le& |\left( \frac{2}{3} \right)^n \sin t| + |\sum_{k=0}^{n-1} \left( \frac{2}{3} \right)^k - 3| \\
		&\le& |\left( \frac{2}{3} \right)^n| + |\sum_{k=0}^{n-1} \left( \frac{2}{3} \right)^k - 3| \quad \text{since \;$\sin t \le 1$ for $t \in \mathbb{R}$} \\
		&\le& \frac{\epsilon}{2} + \frac{\epsilon}{2} \quad \text{for $n \ge N$} \\
		&=& \epsilon
\end{eqnarray*}
So we have shown that $|f_n(t) - 3| < \epsilon$ for $t \in \mathbb{R}$ which gives the uniform convergence of $f_n \to 3$.\\
\indent If we now let $x = t^2$ then $f_n(t) = T_n(t^2)$. If the domain of $f_0(t) = t^2$ is bounded, then $t^2 \le b^2$ where $b$ is the bound. Since $\lim_{n \to \infty} c x_n = c \lim_{n \to \infty} x_n$ where $c$ is a constant we may pick an $N_1$ such that $|\left( \frac{2}{3} \right)^n b^2| < \frac{\epsilon}{2}$ where $n > N_1$. We may perform the same calculation as above to derive:
\begin{eqnarray*}
|f_n(t) - 3| &=& |\left( \frac{2}{3} \right)^n t^2 + \sum_{k=0}^{n-1} \left( \frac{2}{3} \right)^k - 3| \\
		&\le& |\left( \frac{2}{3} \right)^n t^2| + |\sum_{k=0}^{n-1} \left( \frac{2}{3} \right)^k - 3| \\
		&\le& |\left( \frac{2}{3} \right)^n b^2| + |\sum_{k=0}^{n-1} \left( \frac{2}{3} \right)^k - 3| \quad \text{since $t \le b$ for $t \in [a,b]$} \\
		&\le& \frac{\epsilon}{2} + \frac{\epsilon}{2} \quad \text{for $n \ge N$} \\
		&=& \epsilon
\end{eqnarray*}
If the domain of $f_0(t) = t^2$ is $\mathbb{R}$ then we may chose $t'_n = \sqrt{3 \left( \frac{3}{2} \right)^n}$ which gives 
\begin{eqnarray*}
 |f_n(t'_n) - 3| &=& |\left( \frac{2}{3} \right)^n t_n^{'2} + \sum_{k=0}^{n-1} \left( \frac{2}{3} \right)^k - 3| \\
 		      &=& |\left( \frac{2}{3} \right)^n \left( 3 \left( \frac{3}{2} \right)^n \right) + \sum_{k=0}^{n-1} \left( \frac{2}{3} \right)^k - 3| \\
		      &=& | 3 + \sum_{k=0}^{n-1} \left( \frac{2}{3} \right)^k - 3| \\
		      &=& |\sum_{k=0}^{n-1} \left( \frac{2}{3} \right)^k| \ge 1
\end{eqnarray*}
So $ |f_n(t'_n) - 3| \ge 1$ for all $n$ which shows that in the case of the unbounded domain the function sequence can not converge.
\newpage
 
\vspace{15pt}
\begin{flushleft} 
\textbf{Class 18.100B} - Problem 2\\
\rule{500pt}{1pt}\\
\end{flushleft} 

We are given a function $\varphi : [0,\infty) \to \mathbb{R}$ and given that it is continuous satisfies the following:
\[ 0 < \varphi(t) \le \frac{t}{2+t} \quad \text{for $t \in [0,\infty)$} \]
If we know define a sequence by $f_0(t) = \varphi(t)$ and $f_{n+1}(t) = \varphi(f_n(t))$ and let 
\[ F(t) = \sum_{n=0}^\infty f_n(t) \]
then we want to show that $F$ is continuous and converges for every $t \in [0,\infty)$.

\vspace{15pt}
\begin{flushleft} 
\textbf{Class 18.100B} - Problem 3\\
\rule{500pt}{1pt}\\
\end{flushleft} 
 
\vspace{15pt}
\begin{flushleft} 
\textbf{Class 18.100B} - Problem 4\\
\rule{500pt}{1pt}\\
\end{flushleft} 


\vspace{15pt}
\begin{flushleft} 
\textbf{Class 18.100B} - Problem 5\\
\rule{500pt}{1pt}\\
\end{flushleft} 


\vspace{15pt}
\begin{flushleft} 
\textbf{Class 18.100B} - Problem 6\\
\rule{500pt}{1pt}\\
\end{flushleft} 


\vspace{15pt}
\begin{flushleft} 
\textbf{Class 18.100B} - Problem 7\\
\rule{500pt}{1pt}\\
\end{flushleft} 


\end{document}  